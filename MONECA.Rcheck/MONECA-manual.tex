\nonstopmode{}
\documentclass[a4paper]{book}
\usepackage[times,inconsolata,hyper]{Rd}
\usepackage{makeidx}
\makeatletter\@ifl@t@r\fmtversion{2018/04/01}{}{\usepackage[utf8]{inputenc}}\makeatother
% \usepackage{graphicx} % @USE GRAPHICX@
\makeindex{}
\begin{document}
\chapter*{}
\begin{center}
{\textbf{\huge Package `MONECA'}}
\par\bigskip{\large \today}
\end{center}
\ifthenelse{\boolean{Rd@use@hyper}}{\hypersetup{pdftitle = {MONECA: Mobility Network Clustering Analysis}}}{}
\begin{description}
\raggedright{}
\item[Title]\AsIs{Mobility Network Clustering Analysis}
\item[Description]\AsIs{MONECA creates a weighted network from a mobility table and uses cliques (or similar) to create discrete and nested clusters.}
\item[URL]\AsIs{None yet}
\item[Version]\AsIs{0.2.0}
\item[Maintainer]\AsIs{Anton Grau Larsen }\email{agl.ioa@cbs.dk}\AsIs{}
\item[Author]\AsIs{Jonas Touboel }\email{jt@soc.ku.dk}\AsIs{}
\item[Depends]\AsIs{R (>= 3.5.0)}
\item[Imports]\AsIs{ggplot2 (>= 3.4.0), ggraph (>= 2.0.0), grid, igraph (>=
1.3.0), toOrdinal, RColorBrewer (>= 1.1-2), scales (>= 1.2.0),
stats, grDevices, dplyr (>= 1.0.0), tidygraph (>= 1.2.0)}
\item[Suggests]\AsIs{knitr, rmarkdown, testthat (>= 3.0.0)}
\item[License]\AsIs{GPL-3}
\item[RoxygenNote]\AsIs{7.3.2}
\item[VignetteBuilder]\AsIs{knitr}
\item[Encoding]\AsIs{UTF-8}
\item[NeedsCompilation]\AsIs{no}
\end{description}
\Rdcontents{Contents}
\HeaderA{MONECA-package}{MONECA: Mobility Network Clustering Analysis}{MONECA.Rdash.package}
%
\begin{Description}
MONECA creates weighted networks from mobility tables and uses cliques to identify
discrete and nested clusters of social positions with high internal mobility.
The package provides comprehensive tools for analyzing social mobility patterns
through graph-theoretic approaches with modern visualization capabilities.

MONECA creates a weighted network from a mobility table and uses cliques (or similar) to create discrete and nested clusters.
\end{Description}
%
\begin{Details}
The MONECA package implements a sophisticated hierarchical clustering algorithm
that detects mobility patterns in social networks. It identifies segments with
high internal mobility by finding cliques in weighted network representations
of mobility data, creating nested hierarchical structures that reveal the
underlying organization of social mobility.

\strong{Key Features:}
\begin{itemize}

\item{} \strong{Network-based clustering} using relative risk matrices
\item{} \strong{Hierarchical segmentation} with multiple nested levels
\item{} \strong{Modern visualization} with ggraph and ggplot2
\item{} \strong{Synthetic data generation} for testing and demonstrations
\item{} \strong{Quality metrics} for assessing segmentation performance
\item{} \strong{Multiple visualization types} (network, ego, stair plots)
\item{} \strong{Flexible input formats} with automatic data validation

\end{itemize}


\strong{Core Functions:}
\begin{itemize}

\item{} \code{\LinkA{moneca}{moneca}}: Main clustering algorithm
\item{} \code{\LinkA{find.segments}{find.segments}}: Core segmentation identification
\item{} \code{\LinkA{weight.matrix}{weight.matrix}}: Relative risk matrix calculation
\item{} \code{\LinkA{segment.membership}{segment.membership}}: Extract cluster memberships

\end{itemize}


\strong{Modern Visualization (Recommended):}
\begin{itemize}

\item{} \code{\LinkA{plot\_moneca\_ggraph}{plot.Rul.moneca.Rul.ggraph}}: Network visualization with ggraph
\item{} \code{\LinkA{plot\_ego\_ggraph}{plot.Rul.ego.Rul.ggraph}}: Ego network analysis
\item{} \code{\LinkA{plot\_stair\_ggraph}{plot.Rul.stair.Rul.ggraph}}: Multi-level segmentation plots

\end{itemize}


\strong{Legacy Visualization (Backward Compatibility):}
\begin{itemize}

\item{} \code{\LinkA{gg.moneca}{gg.moneca}}: Original ggplot2-based visualization
\item{} \code{\LinkA{moneca.plot}{moneca.plot}}: Base R network plotting
\item{} \code{\LinkA{ego.plot}{ego.plot}}: Legacy ego network plots
\item{} \code{\LinkA{stair.plot}{stair.plot}}: Legacy stair plots

\end{itemize}


\strong{Data Generation:}
\begin{itemize}

\item{} \code{\LinkA{generate\_mobility\_data}{generate.Rul.mobility.Rul.data}}: Create synthetic mobility tables
\item{} \code{\LinkA{generate\_example\_datasets}{generate.Rul.example.Rul.datasets}}: Predefined example datasets

\end{itemize}

\end{Details}
%
\begin{Section}{Package Dependencies}

\strong{Required packages:}
\begin{itemize}

\item{} \code{igraph} (>= 1.3.0): Network analysis with compatibility layer
\item{} \code{ggplot2} (>= 3.4.0): Graphics and plotting
\item{} \code{RColorBrewer}: Color palettes for visualizations
\item{} \code{scales}: Scale transformations and formatting

\end{itemize}


\strong{Modern visualization packages:}
\begin{itemize}

\item{} \code{ggraph} (>= 2.0.0): Advanced network visualization
\item{} \code{tidygraph} (>= 1.2.0): Tidy graph manipulation
\item{} \code{dplyr} (>= 1.0.0): Data manipulation

\end{itemize}


\strong{Additional packages:}
\begin{itemize}

\item{} \code{grid}: Low-level graphics
\item{} \code{toOrdinal}: Ordinal number formatting

\end{itemize}

\end{Section}
%
\begin{Author}
Jonas Touboel \email{jonas.touboel@ku.dk}

Anton Grau Larsen \email{agl.ioa@cbs.dk} (maintainer)
\end{Author}
%
\begin{References}
Touboel, J., \& Larsen, A. G. (2017). Mapping the Social Class Structure: 
From Occupational Mobility to Social Class Categories Using Network Analysis.
\emph{Sociology}, 51(6), 1257-1276. \Rhref{https://doi.org/10.1177/0038038516673125}{doi:10.1177\slash{}0038038516673125}

Larsen, A. G., \& Touboel, J. (2015). Social network analysis of occupational 
mobility and social class. In \emph{Research Handbook on Analytical Sociology} 
(pp. 411-434). Edward Elgar Publishing.
\end{References}
%
\begin{SeeAlso}
Useful links:
\begin{itemize}

\item{} \url{None yet}

\end{itemize}



Useful links:
\begin{itemize}

\item{} \url{None yet}

\end{itemize}


\end{SeeAlso}
%
\begin{Examples}
\begin{ExampleCode}
# Generate synthetic mobility data
mobility_data <- generate_mobility_data(n_classes = 6, seed = 42)

# Run MONECA analysis
seg <- moneca(mobility_data, segment.levels = 3)
print(seg)

# Extract segment membership
membership <- segment.membership(seg)
print(membership)

# Modern visualization (recommended)
## Not run: 
# Basic network plot
plot_moneca_ggraph(seg, title = "Social Mobility Network")

# Customized network visualization
plot_moneca_ggraph(seg, 
                  layout = "stress",
                  node_color = "mobility",
                  edge_width = "weight",
                  color_palette = "Spectral")

# Ego network analysis
plot_ego_ggraph(seg, mobility_data, ego_id = 3, 
               title = "Mobility from Middle Class")

# Multi-level stair plot
plot_stair_ggraph(seg)

## End(Not run)

# Use predefined examples
examples <- generate_example_datasets()
simple_seg <- moneca(examples$simple, segment.levels = 2)

## Not run: 
# Compare different mobility regimes
plot_moneca_ggraph(simple_seg, title = "Simple Mobility Structure")

## End(Not run)

\end{ExampleCode}
\end{Examples}
\HeaderA{.moneca\_env}{igraph Compatibility Layer}{.moneca.Rul.env}
\keyword{internal}{.moneca\_env}
%
\begin{Description}
This file provides compatibility functions to handle both old and new igraph API
ensuring MONECA works with igraph versions from 1.3.0 onwards
\end{Description}
%
\begin{Usage}
\begin{verbatim}
.moneca_env
\end{verbatim}
\end{Usage}
%
\begin{Format}
An object of class \code{environment} of length 7.
\end{Format}
\HeaderA{analyze\_package\_dependencies}{Analyze package dependencies usage}{analyze.Rul.package.Rul.dependencies}
%
\begin{Description}
Analyze package dependencies usage
\end{Description}
%
\begin{Usage}
\begin{verbatim}
analyze_package_dependencies()
\end{verbatim}
\end{Usage}
%
\begin{Value}
Data frame with dependency analysis
\end{Value}
\HeaderA{apply\_pattern\_rules}{Apply Pattern Rules for Segment Naming}{apply.Rul.pattern.Rul.rules}
\keyword{internal}{apply\_pattern\_rules}
%
\begin{Description}
Internal helper function that applies pattern recognition rules to generate
meaningful segment names based on common suffixes or prefixes.
\end{Description}
%
\begin{Usage}
\begin{verbatim}
apply_pattern_rules(node_names, pattern_rules, membership_id)
\end{verbatim}
\end{Usage}
%
\begin{Arguments}
\begin{ldescription}
\item[\code{node\_names}] Character vector of node names in the segment

\item[\code{pattern\_rules}] Named list of pattern replacement rules

\item[\code{membership\_id}] Character string of membership ID for fallback naming
\end{ldescription}
\end{Arguments}
%
\begin{Value}
Character string with pattern-based name, or NULL if no pattern found
\end{Value}
\HeaderA{ego.plot}{Legacy Ego Network Visualization}{ego.plot}
%
\begin{Description}
Creates ego network plots showing mobility patterns from a single focal position
using the legacy ggplot2 plotting system. For modern ego network analysis,
use \code{\LinkA{plot\_ego\_ggraph}{plot.Rul.ego.Rul.ggraph}}.
\end{Description}
%
\begin{Usage}
\begin{verbatim}
ego.plot(
  segments,
  mxa.b,
  id = 1,
  lay = NULL,
  edge.size = 0.8,
  border.padding = 1,
  title.line = TRUE,
  vertex.size = "totals",
  small.cell.reduction = 5,
  edge.weight = "discrete",
  color.scheme = "RdPu",
  ...
)
\end{verbatim}
\end{Usage}
%
\begin{Arguments}
\begin{ldescription}
\item[\code{segments}] A MONECA object returned by \code{\LinkA{moneca}{moneca}}.

\item[\code{mxa.b}] The original mobility matrix used in the MONECA analysis.

\item[\code{id}] Integer or character specifying the focal node (ego) for the analysis.

\item[\code{lay}] Layout matrix for node positioning, typically from \code{\LinkA{layout.matrix}{layout.matrix}}.

\item[\code{edge.size}] Numeric value for edge thickness. Default is 0.8.

\item[\code{border.padding}] Numeric value for segment boundary padding. Default is 1.

\item[\code{title.line}] Logical indicating whether to add a title line. Default is TRUE.

\item[\code{vertex.size}] Specification for vertex sizes. If "totals", sizes are derived
from row/column totals in the mobility matrix. Otherwise, uses the specified values.

\item[\code{small.cell.reduction}] Numeric threshold below which vertices receive different
shapes to indicate low mobility volumes. Default is 5.

\item[\code{edge.weight}] Character string specifying edge weight display. "discrete"
creates categorical edge weights, otherwise uses continuous weights.

\item[\code{color.scheme}] Character string specifying the RColorBrewer color scheme
for the visualization. Default is "RdPu".

\item[\code{...}] Additional arguments passed to \code{\LinkA{gg.moneca}{gg.moneca}}.
\end{ldescription}
\end{Arguments}
%
\begin{Details}
This function creates a focused view of mobility patterns from a single position
in the social structure. It highlights both incoming and outgoing mobility flows
and uses different visual elements to represent:
\begin{itemize}

\item{} Edge colors/weights for relative risk levels
\item{} Node sizes for total mobility volumes
\item{} Node shapes for positions with low mobility
\item{} Node colors for mobility share proportions

\end{itemize}


\strong{Note}: This function is maintained for backward compatibility.
For new analyses, consider using \code{\LinkA{plot\_ego\_ggraph}{plot.Rul.ego.Rul.ggraph}} which offers
better performance and more modern styling options.
\end{Details}
%
\begin{Value}
A ggplot2 object showing the ego network.
\end{Value}
%
\begin{SeeAlso}
\code{\LinkA{plot\_ego\_ggraph}{plot.Rul.ego.Rul.ggraph}} for modern ego network visualization,
\code{\LinkA{gg.moneca}{gg.moneca}} for the underlying plotting function
\end{SeeAlso}
%
\begin{Examples}
\begin{ExampleCode}
## Not run: 
# Requires legacy data and eliter package
data(occupations)
ego.plot(mob.seg, mob.mat, id = 2)

# Customized ego plot
ego.plot(mob.seg, mob.mat, 
         id = 3,
         edge.size = 1.2,
         color.scheme = "Blues",
         small.cell.reduction = 10)

## End(Not run)

\end{ExampleCode}
\end{Examples}
\HeaderA{find.segments}{Find Segments in Mobility Networks}{find.segments}
%
\begin{Description}
Identifies discrete groups or segments based on a weighted network matrix using a 
clique-based algorithm. This function implements the core segmentation algorithm
that iteratively assigns nodes to segments based on network ties.
\end{Description}
%
\begin{Usage}
\begin{verbatim}
find.segments(
  mat,
  cliques,
  cut.off = 1,
  mode = "symmetric",
  delete.upper.tri = TRUE
)
\end{verbatim}
\end{Usage}
%
\begin{Arguments}
\begin{ldescription}
\item[\code{mat}] A weighted adjacency matrix representing mobility flows or relationships.
Should include row and column names representing the categories/classes.

\item[\code{cliques}] A list of cliques (complete subgraphs) in the network, typically
obtained from \code{igraph::cliques()}.

\item[\code{cut.off}] Numeric threshold for minimum weight or relative risk to be 
considered a network tie. Default is 1.

\item[\code{mode}] Character string specifying how to handle asymmetric relationships:
\begin{itemize}

\item{} "symmetric" (default): Standard symmetric treatment
\item{} "Mutual": Only mutual ties (bidirectional) are considered
\item{} "Unmutual": Unidirectional ties are allowed

\end{itemize}


\item[\code{delete.upper.tri}] Logical indicating whether to process only the lower 
triangle of the matrix for efficiency. Default is TRUE.
\end{ldescription}
\end{Arguments}
%
\begin{Details}
The algorithm works by iteratively examining the strongest remaining ties in the
network and assigning nodes to segments based on clique membership. It uses a
greedy approach that prioritizes stronger connections and larger existing segments.
\end{Details}
%
\begin{Value}
A list with two components:
\begin{description}

\item[membership] A factor indicating segment membership for each node
\item[cliques] A list where each element contains the indices of nodes 
belonging to that segment

\end{description}

\end{Value}
%
\begin{SeeAlso}
\code{\LinkA{moneca}{moneca}} for the main analysis function
\end{SeeAlso}
\HeaderA{find\_common\_prefix}{Find Common Prefix in Node Names}{find.Rul.common.Rul.prefix}
\keyword{internal}{find\_common\_prefix}
%
\begin{Description}
Find Common Prefix in Node Names
\end{Description}
%
\begin{Usage}
\begin{verbatim}
find_common_prefix(names)
\end{verbatim}
\end{Usage}
%
\begin{Arguments}
\begin{ldescription}
\item[\code{names}] Character vector of names
\end{ldescription}
\end{Arguments}
%
\begin{Value}
Character string of common prefix, or empty string if none
\end{Value}
\HeaderA{find\_common\_suffix}{Find Common Suffix in Node Names}{find.Rul.common.Rul.suffix}
\keyword{internal}{find\_common\_suffix}
%
\begin{Description}
Find Common Suffix in Node Names
\end{Description}
%
\begin{Usage}
\begin{verbatim}
find_common_suffix(names)
\end{verbatim}
\end{Usage}
%
\begin{Arguments}
\begin{ldescription}
\item[\code{names}] Character vector of names
\end{ldescription}
\end{Arguments}
%
\begin{Value}
Character string of common suffix, or empty string if none
\end{Value}
\HeaderA{first.level.summary}{First level summary}{first.level.summary}
%
\begin{Description}
First level summary
\end{Description}
%
\begin{Usage}
\begin{verbatim}
first.level.summary(
  segments,
  small.cell.reduction = segments$small.cell.reduction
)
\end{verbatim}
\end{Usage}
\HeaderA{force.segments}{force.segments}{force.segments}
%
\begin{Description}
Create a two-level segment-object with a forced solutionz
\end{Description}
%
\begin{Usage}
\begin{verbatim}
force.segments(segments, variable)
\end{verbatim}
\end{Usage}
\HeaderA{generate\_dependency\_report}{Generate comprehensive dependency report}{generate.Rul.dependency.Rul.report}
%
\begin{Description}
Generate comprehensive dependency report
\end{Description}
%
\begin{Usage}
\begin{verbatim}
generate_dependency_report(output_file = NULL)
\end{verbatim}
\end{Usage}
%
\begin{Arguments}
\begin{ldescription}
\item[\code{output\_file}] Optional file path to save the report
\end{ldescription}
\end{Arguments}
%
\begin{Value}
Invisible NULL
\end{Value}
\HeaderA{generate\_example\_datasets}{Generate Predefined Example Datasets}{generate.Rul.example.Rul.datasets}
%
\begin{Description}
Creates a collection of predefined mobility datasets with different structural
characteristics. These examples are useful for tutorials, testing different
MONECA parameters, and demonstrating various mobility regime types.
\end{Description}
%
\begin{Usage}
\begin{verbatim}
generate_example_datasets()
\end{verbatim}
\end{Usage}
%
\begin{Details}
This function provides ready-to-use examples that demonstrate different types
of social mobility structures:

\begin{itemize}

\item{} \strong{Simple}: Good for learning MONECA basics
\item{} \strong{Complex}: Tests algorithm performance on larger structures
\item{} \strong{Rigid}: Shows segmentation in low-mobility societies
\item{} \strong{Fluid}: Tests segmentation in high-mobility contexts
\item{} \strong{Polarized}: Demonstrates class divide patterns
\item{} \strong{Gradual}: Shows continuous mobility gradients

\end{itemize}


These examples are particularly useful for:
\begin{itemize}

\item{} Tutorial and teaching materials
\item{} Comparing MONECA parameters across mobility types
\item{} Testing visualization functions
\item{} Benchmarking algorithm performance

\end{itemize}

\end{Details}
%
\begin{Value}
A named list containing mobility matrices with different characteristics:
\begin{description}

\item[simple] Small 4-class example with clear segmentation
\item[complex] Larger 8-class example with hierarchical structure
\item[rigid] High immobility example showing strong class boundaries
\item[fluid] Low immobility example with extensive mobility
\item[polarized] Example with strong upper/lower class divide
\item[gradual] Example with smooth class transitions

\end{description}

Each matrix is in standard MONECA format with row/column totals.
\end{Value}
%
\begin{SeeAlso}
\code{\LinkA{generate\_mobility\_data}{generate.Rul.mobility.Rul.data}} for custom synthetic data,
\code{\LinkA{moneca}{moneca}} for analysis,
\code{\LinkA{plot\_moneca\_ggraph}{plot.Rul.moneca.Rul.ggraph}} for visualization
\end{SeeAlso}
%
\begin{Examples}
\begin{ExampleCode}
# Load all example datasets
examples <- generate_example_datasets()
names(examples)

# Examine the simple example
print(examples$simple)

# Compare different mobility regimes
rigid_seg <- moneca(examples$rigid, segment.levels = 3)
fluid_seg <- moneca(examples$fluid, segment.levels = 3)

# Visualize different structures
## Not run: 
plot_moneca_ggraph(rigid_seg, title = "Rigid Class Structure")
plot_moneca_ggraph(fluid_seg, title = "Fluid Mobility Regime")

# Create comparative stair plots
stair_rigid <- plot_stair_ggraph(rigid_seg)
stair_fluid <- plot_stair_ggraph(fluid_seg)

## End(Not run)

# Analyze segmentation quality across examples
for (name in names(examples)) {
  seg <- moneca(examples[[name]], segment.levels = 2)
  cat("Dataset:", name, "- Segments:", length(seg$segment.list[[2]]), "\n")
}

\end{ExampleCode}
\end{Examples}
\HeaderA{generate\_level\_name}{Generate Level Name Based on Strategy}{generate.Rul.level.Rul.name}
\keyword{internal}{generate\_level\_name}
%
\begin{Description}
Internal helper function to generate meaningful level names for segments.
\end{Description}
%
\begin{Usage}
\begin{verbatim}
generate_level_name(
  node_names,
  membership_id,
  naming_strategy,
  custom_names,
  separator,
  max_concat_length,
  pattern_rules
)
\end{verbatim}
\end{Usage}
%
\begin{Arguments}
\begin{ldescription}
\item[\code{node\_names}] Character vector of node names in the segment

\item[\code{membership\_id}] Character string of the membership ID (e.g., "2.1")

\item[\code{naming\_strategy}] Character string specifying naming approach

\item[\code{custom\_names}] Named list of custom names

\item[\code{separator}] Character string for concatenation

\item[\code{max\_concat\_length}] Maximum nodes to concatenate

\item[\code{pattern\_rules}] Named list of pattern replacement rules
\end{ldescription}
\end{Arguments}
%
\begin{Value}
Character string with the generated level name
\end{Value}
\HeaderA{generate\_mobility\_data}{Generate Realistic Synthetic Mobility Data}{generate.Rul.mobility.Rul.data}
%
\begin{Description}
Creates synthetic social mobility tables with configurable patterns that mimic
real-world mobility structures. This function is essential for testing MONECA
algorithms, creating examples, and understanding how different mobility patterns
affect segmentation results.
\end{Description}
%
\begin{Usage}
\begin{verbatim}
generate_mobility_data(
  n_classes = 10,
  n_total = 10000,
  immobility_strength = 0.6,
  class_clustering = 0.3,
  noise_level = 0.1,
  class_names = NULL,
  seed = NULL
)
\end{verbatim}
\end{Usage}
%
\begin{Arguments}
\begin{ldescription}
\item[\code{n\_classes}] Integer specifying the number of social classes or occupational
categories. Must be at least 2. Default is 10.

\item[\code{n\_total}] Integer specifying the total sample size (number of individuals).
Must be at least as large as n\_classes. Default is 10000.

\item[\code{immobility\_strength}] Numeric value (0-1) controlling the strength of
diagonal immobility (tendency to remain in origin class). Higher values create
more rigid class structures. Default is 0.6.

\item[\code{class\_clustering}] Numeric value (0-1) controlling the tendency for mobility
between adjacent classes in the social hierarchy. Higher values create more
gradual, short-distance mobility. Default is 0.3.

\item[\code{noise\_level}] Numeric value (0-1) controlling the amount of random mobility
between non-adjacent classes. Higher values create more chaotic mobility patterns.
Default is 0.1.

\item[\code{class\_names}] Character vector of names for the social classes. If NULL,
classes are named "Class 1", "Class 2", etc. Length must equal n\_classes.

\item[\code{seed}] Integer for random seed to ensure reproducible results. Default is NULL
(no seed set).
\end{ldescription}
\end{Arguments}
%
\begin{Details}
This function generates mobility data using a hierarchical model that reflects
common patterns in social mobility research:

\begin{enumerate}

\item{} \strong{Diagonal Dominance}: Most people remain in their origin class
\item{} \strong{Distance Effects}: Mobility is more likely between adjacent classes
\item{} \strong{Marginal Heterogeneity}: Class sizes vary realistically
\item{} \strong{Random Component}: Some unpredictable mobility occurs

\end{enumerate}


The generated data can be used to test how different mobility regimes affect
MONECA segmentation and to create realistic examples for demonstrations.

Parameter combinations:
\begin{itemize}

\item{} High immobility + low clustering = rigid class boundaries
\item{} Low immobility + high clustering = fluid but structured mobility
\item{} High noise = chaotic mobility patterns
\item{} Balanced parameters = realistic social mobility

\end{itemize}

\end{Details}
%
\begin{Value}
A square matrix in MONECA format with:
\begin{description}

\item[Core matrix] Upper-left (n\_classes x n\_classes) contains mobility flows
\item[Row margins] Last column contains origin class totals
\item[Column margins] Last row contains destination class totals
\item[Grand total] Bottom-right cell contains total sample size

\end{description}

The matrix includes row and column names for easy interpretation.
\end{Value}
%
\begin{SeeAlso}
\code{\LinkA{moneca}{moneca}} for analyzing the generated data,
\code{\LinkA{generate\_example\_datasets}{generate.Rul.example.Rul.datasets}} for predefined examples,
\code{\LinkA{plot\_moneca\_ggraph}{plot.Rul.moneca.Rul.ggraph}} for visualizing results
\end{SeeAlso}
%
\begin{Examples}
\begin{ExampleCode}
# Basic synthetic data with default parameters
basic_data <- generate_mobility_data()
print(basic_data[1:6, 1:6])  # Show first 5 classes plus totals

# Small example for quick testing
test_data <- generate_mobility_data(
  n_classes = 5,
  n_total = 1000,
  immobility_strength = 0.7,
  class_clustering = 0.2,
  seed = 42
)

# Professional class structure with custom names
professional_data <- generate_mobility_data(
  n_classes = 6,
  n_total = 5000,
  class_names = c("Upper", "Upper-Middle", "Middle", 
                  "Lower-Middle", "Working", "Lower"),
  immobility_strength = 0.5,
  class_clustering = 0.4,
  seed = 123
)

# Highly fluid mobility regime
fluid_data <- generate_mobility_data(
  n_classes = 8,
  immobility_strength = 0.3,
  class_clustering = 0.1,
  noise_level = 0.3
)

# Use in MONECA analysis
seg <- moneca(test_data, segment.levels = 3)
plot_moneca_ggraph(seg, title = "Synthetic Mobility Analysis")

\end{ExampleCode}
\end{Examples}
\HeaderA{gg.moneca}{Legacy ggplot2 Visualization for MONECA Objects}{gg.moneca}
%
\begin{Description}
Creates network visualizations of MONECA clustering results using ggplot2.
This function provides extensive customization options but has been superseded
by \code{\LinkA{plot\_moneca\_ggraph}{plot.Rul.moneca.Rul.ggraph}} for most use cases.
\end{Description}
%
\begin{Usage}
\begin{verbatim}
gg.moneca(
  segments,
  level = NULL,
  layout = NULL,
  edges = NULL,
  mode = "directed",
  vertex.size = "total",
  vertex.fill = "segment",
  vertex.alpha = 1,
  vertex.color = "black",
  vertex.shape = 21,
  show.edges = TRUE,
  edge.size = 1,
  edge.alpha = "weight",
  edge.color = "weight",
  edge.line = "solid",
  show.text = TRUE,
  text.size = 3,
  text.color = "black",
  text.alpha = 1,
  text.vjust = 1.5,
  show.borders = TRUE,
  border.size = 1,
  border.fill = NA,
  border.color = "black",
  border.alpha = 1,
  border.padding = 0.7,
  border.text = TRUE,
  border.labels = "segments",
  border.text.size = 4,
  border.text.color = "black",
  border.text.vjust = -0.2,
  border.text.hjust = 1,
  midpoints = TRUE,
  midpoint.arrow = arrow(angle = 20, length = unit(0.33, "cm"), ends = "last", type =
    "closed"),
  edge.text = FALSE,
  edge.text.size = 3,
  edge.text.alpha = 0.9,
  legend = "side"
)
\end{verbatim}
\end{Usage}
%
\begin{Arguments}
\begin{ldescription}
\item[\code{segments}] A MONECA object returned by \code{\LinkA{moneca}{moneca}}.

\item[\code{level}] Integer vector specifying which hierarchical levels to display.

\item[\code{layout}] Matrix of node coordinates or layout function result.

\item[\code{edges}] Edge matrix or transformed edge weights for visualization.

\item[\code{mode}] Character string specifying graph mode ("directed" or "undirected").

\item[\code{vertex.size}] Aesthetic for vertex size. Can be "total", "mobility", or numeric.

\item[\code{vertex.fill}] Aesthetic for vertex fill color. Can be "segment" or color specification.

\item[\code{vertex.alpha}] Numeric value (0-1) for vertex transparency.

\item[\code{vertex.color}] Color for vertex borders.

\item[\code{vertex.shape}] Numeric code for vertex shape (see ggplot2 shapes).

\item[\code{show.edges}] Logical indicating whether to display edges.

\item[\code{edge.size}] Size specification for edges.

\item[\code{edge.alpha}] Transparency for edges. Can be "weight" or numeric.

\item[\code{edge.color}] Color specification for edges. Can be "weight" or color name.

\item[\code{edge.line}] Line type for edges ("solid", "dashed", etc.).

\item[\code{show.text}] Logical indicating whether to show vertex labels.

\item[\code{text.size}] Numeric size for vertex labels.

\item[\code{text.color}] Color for vertex labels.

\item[\code{text.alpha}] Transparency for vertex labels.

\item[\code{text.vjust}] Vertical adjustment for labels.

\item[\code{show.borders}] Logical indicating whether to show segment boundaries.

\item[\code{border.size}] Size for segment borders.

\item[\code{border.fill}] Fill color for segment boundaries.

\item[\code{border.color}] Color for segment border lines.

\item[\code{border.alpha}] Transparency for segment borders.

\item[\code{border.padding}] Padding around segment boundaries.

\item[\code{border.text}] Logical indicating whether to show segment labels.

\item[\code{border.labels}] Character vector of custom segment labels.

\item[\code{border.text.size}] Size for segment labels.

\item[\code{border.text.color}] Color for segment labels.

\item[\code{border.text.vjust}] Vertical adjustment for segment labels.

\item[\code{border.text.hjust}] Horizontal adjustment for segment labels.

\item[\code{midpoints}] Logical indicating whether to show edge midpoints.

\item[\code{midpoint.arrow}] Logical indicating whether to show arrows at midpoints.

\item[\code{edge.text}] Logical indicating whether to show edge labels.

\item[\code{edge.text.size}] Size for edge labels.

\item[\code{edge.text.alpha}] Transparency for edge labels.

\item[\code{legend}] Position for legend ("side", "bottom", "none", etc.).
\end{ldescription}
\end{Arguments}
%
\begin{Details}
This function provides a highly customizable but complex interface for creating
MONECA visualizations. It requires the eliter package for some functionality.
For most users, \code{\LinkA{plot\_moneca\_ggraph}{plot.Rul.moneca.Rul.ggraph}} offers a more modern and
user-friendly interface with better defaults.

\strong{Note}: This function is maintained for backward compatibility but
is no longer actively developed. New features are added to the ggraph-based
plotting functions instead.
\end{Details}
%
\begin{Value}
A ggplot2 object.
\end{Value}
%
\begin{SeeAlso}
\code{\LinkA{plot\_moneca\_ggraph}{plot.Rul.moneca.Rul.ggraph}} for modern ggraph-based plotting,
\code{\LinkA{moneca.plot}{moneca.plot}} for base graphics plotting
\end{SeeAlso}
%
\begin{Examples}
\begin{ExampleCode}
## Not run: 
# Requires eliter package
data(occupations)
gg.moneca(mob.seg)

# Custom styling
gg.moneca(mob.seg, 
         vertex.fill = "red",
         edge.color = "blue",
         show.borders = FALSE)

## End(Not run)

\end{ExampleCode}
\end{Examples}
\HeaderA{layout.matrix}{Layout matrix}{layout.matrix}
%
\begin{Description}
A matrix with the coordinates of the segments
\end{Description}
%
\begin{Usage}
\begin{verbatim}
layout.matrix(
  segments,
  attraction = c(320, 40, 10, 4, 2),
  level = seq(segments$segment.list),
  mode = "directed",
  weight.adjustment = 1,
  start.temp = 20,
  niter = 10000,
  tie.adjustment = 0.4,
  ...
)
\end{verbatim}
\end{Usage}
%
\begin{Arguments}
\begin{ldescription}
\item[\code{segments}] a segment object

\item[\code{attraction}] the distance between the segment points for each level.

\item[\code{level}] the included levels

\item[\code{mode}] the mode

\item[\code{area.size}] the size of the plot area - see \LinkA{layout.fruchterman.reingold}{layout.fruchterman.reingold}
\end{ldescription}
\end{Arguments}
\HeaderA{modern\_plotting}{Modern MONECA Plotting with ggraph}{modern.Rul.plotting}
%
\begin{Description}
Enhanced plotting functions using ggplot2 and ggraph for MONECA objects.
These functions replace the old plotting system with modern, customizable
network visualizations.
\end{Description}
\HeaderA{moneca}{MONECA - Mobility Network Clustering Analysis}{moneca}
%
\begin{Description}
Main function for performing hierarchical clustering analysis on mobility tables.
MONECA creates weighted networks from mobility data and uses cliques to identify
discrete and nested clusters of positions with high internal mobility.
\end{Description}
%
\begin{Usage}
\begin{verbatim}
moneca(
  mx = mx,
  segment.levels = 3,
  cut.off = 1,
  mode = "symmetric",
  delete.upper.tri = TRUE,
  small.cell.reduction = 0
)
\end{verbatim}
\end{Usage}
%
\begin{Arguments}
\begin{ldescription}
\item[\code{mx}] A mobility table (square matrix) with row and column totals in the last
row/column. Row names should identify the categories/classes.

\item[\code{segment.levels}] Integer specifying the number of hierarchical segmentation 
levels to compute. Default is 3. The algorithm may return fewer levels if no
further meaningful segmentation is possible.

\item[\code{cut.off}] Numeric threshold for the minimum relative risk to be considered
a significant tie. Default is 1 (no mobility above random expectation required).

\item[\code{mode}] Character string specifying edge mode ("symmetric", "Mutual", or 
"Unmutual"). Currently not fully implemented - uses symmetric mode.

\item[\code{delete.upper.tri}] Logical indicating whether to use only lower triangle 
for efficiency. Default is TRUE.

\item[\code{small.cell.reduction}] Numeric value to handle small cell counts. Cells with
counts below this threshold are set to 0. Default is 0 (no reduction).
\end{ldescription}
\end{Arguments}
%
\begin{Details}
MONECA implements an iterative algorithm that:
\begin{enumerate}

\item{} Converts the mobility table to a relative risk matrix
\item{} Identifies network cliques based on the threshold
\item{} Groups nodes into segments using the clique structure
\item{} Aggregates the mobility table by segments
\item{} Repeats the process for the specified number of levels

\end{enumerate}


The algorithm stops early if no further segmentation is possible (e.g., all
nodes collapse into a single segment).
\end{Details}
%
\begin{Value}
An object of class "moneca" containing:
\begin{description}

\item[segment.list] A list of segment memberships for each hierarchical level.
Each element is a list of vectors containing the original row indices.
\item[mat.list] A list of aggregated mobility matrices for each level, where
rows/columns represent segments instead of original categories.
\item[small.cell.reduction] The small cell reduction parameter used.

\end{description}

\end{Value}
%
\begin{References}
Toubøl, J., \& Larsen, A. G. (2017). Mapping the Social Class Structure: 
From Occupational Mobility to Social Class Categories Using Network Analysis.
Sociology, 51(6), 1257-1276.
\end{References}
%
\begin{SeeAlso}
\code{\LinkA{find.segments}{find.segments}} for the core segmentation algorithm,
\code{\LinkA{weight.matrix}{weight.matrix}} for relative risk calculation,
\code{\LinkA{plot\_moneca\_ggraph}{plot.Rul.moneca.Rul.ggraph}} for modern visualization,
\code{\LinkA{segment.membership}{segment.membership}} for extracting memberships
\end{SeeAlso}
%
\begin{Examples}
\begin{ExampleCode}
# Generate synthetic mobility data
mobility_data <- generate_mobility_data(n_classes = 6, seed = 42)

# Run MONECA analysis
seg <- moneca(mobility_data, segment.levels = 3)
print(seg)

# Examine segment membership
membership <- segment.membership(seg)
print(membership)

# Visualize with modern plotting
## Not run: 
plot_moneca_ggraph(seg, node_color = "segment", title = "MONECA Clustering")

## End(Not run)

\end{ExampleCode}
\end{Examples}
\HeaderA{moneca.plot}{Legacy Network Plot for MONECA Results}{moneca.plot}
%
\begin{Description}
Creates a network visualization of MONECA segmentation results using base
graphics and igraph. For modern visualizations, use \code{\LinkA{plot\_moneca\_ggraph}{plot.Rul.moneca.Rul.ggraph}}.
\end{Description}
%
\begin{Usage}
\begin{verbatim}
moneca.plot(
  segments,
  layout = NULL,
  edges = NULL,
  mode = "directed",
  level = NULL,
  vertex.size = 5,
  vertex.frame.color = "black",
  edge.curved = FALSE,
  vertex.color = "grey50",
  vertex.label.color = "black",
  vertex.label.cex = 0.5,
  vertex.label.dist = 0.12,
  edge.arrow.size = 0.1,
  mark.col = NULL,
  mark.expand = 10,
  border.col = "black",
  edge.width = 1,
  edge.color = "black"
)
\end{verbatim}
\end{Usage}
%
\begin{Arguments}
\begin{ldescription}
\item[\code{segments}] A MONECA object returned by \code{\LinkA{moneca}{moneca}}.

\item[\code{layout}] A matrix of node coordinates, typically from \code{\LinkA{layout.matrix}{layout.matrix}}.

\item[\code{edges}] An adjacency matrix of edges, typically from \code{\LinkA{segment.edges}{segment.edges}}.

\item[\code{mode}] Character string specifying graph mode ("directed" or "undirected").

\item[\code{level}] Integer vector of hierarchical levels to visualize.

\item[\code{vertex.size}] Numeric value for vertex size. Default is 5.

\item[\code{vertex.frame.color}] Color for vertex borders. Default is "black".

\item[\code{edge.curved}] Logical for curved edges. Default is FALSE.

\item[\code{vertex.color}] Color for vertices. Default is "grey50".

\item[\code{vertex.label.color}] Color for vertex labels. Default is "black".

\item[\code{vertex.label.cex}] Size multiplier for labels. Default is 0.5.

\item[\code{vertex.label.dist}] Distance of labels from vertices. Default is 0.12.

\item[\code{edge.arrow.size}] Size of edge arrows. Default is 0.1.

\item[\code{mark.col}] Color for segment markers. Default is NULL.

\item[\code{mark.expand}] Expansion factor for segment boundaries. Default is 10.

\item[\code{border.col}] Color for segment borders. Default is "black".

\item[\code{edge.width}] Width of edges. Default is 1.

\item[\code{edge.color}] Color for edges. Can be a color name or matrix. Default is "black".
\end{ldescription}
\end{Arguments}
%
\begin{Details}
This function provides backward compatibility with earlier versions of MONECA.
For new analyses, consider using \code{\LinkA{plot\_moneca\_ggraph}{plot.Rul.moneca.Rul.ggraph}} which offers
more modern styling and customization options.
\end{Details}
%
\begin{Value}
NULL (creates a plot as side effect).
\end{Value}
%
\begin{SeeAlso}
\code{\LinkA{plot\_moneca\_ggraph}{plot.Rul.moneca.Rul.ggraph}} for modern plotting
\end{SeeAlso}
\HeaderA{occupations}{Occupational mobility}{occupations}
%
\begin{Description}
Occupational mobility
\end{Description}
%
\begin{Examples}
\begin{ExampleCode}
data(occupations)


\end{ExampleCode}
\end{Examples}
\HeaderA{plot\_ego\_ggraph}{Ego Network Visualization with ggraph}{plot.Rul.ego.Rul.ggraph}
%
\begin{Description}
Creates focused visualizations of mobility patterns from a single focal node
(ego). This function shows all incoming and outgoing mobility flows for a
specific category, making it ideal for understanding individual position dynamics.
\end{Description}
%
\begin{Usage}
\begin{verbatim}
plot_ego_ggraph(
  segments,
  mobility_matrix,
  ego_id,
  min_weight = 0,
  layout = "stress",
  highlight_color = "red",
  flow_color = "viridis",
  node_size_range = c(2, 8),
  edge_width_range = c(0.2, 3),
  title = NULL,
  segment_naming = "auto",
  ...
)
\end{verbatim}
\end{Usage}
%
\begin{Arguments}
\begin{ldescription}
\item[\code{segments}] A MONECA object returned by \code{\LinkA{moneca}{moneca}}.

\item[\code{mobility\_matrix}] The original mobility matrix used in the MONECA analysis.
Should include row and column totals.

\item[\code{ego\_id}] Integer or character specifying the focal node. Can be:
\begin{itemize}

\item{} Integer: Row/column index in the mobility matrix
\item{} Character: Row/column name from the mobility matrix

\end{itemize}


\item[\code{min\_weight}] Numeric threshold for minimum edge weight to include nodes.
Only nodes connected to the ego with edge weights >= min\_weight will be shown.
Default is 0 (show all connected nodes). Use higher values to focus on 
stronger mobility flows.

\item[\code{layout}] Character string specifying the layout algorithm. Default is "stress"
which often works well for ego networks. Other options include "fr", "kk", "dh".

\item[\code{highlight\_color}] Color for the ego (focal) node. Default is "red".

\item[\code{flow\_color}] Character string specifying the color scheme for mobility flows.
Default is "viridis". Can be any viridis variant ("viridis", "plasma", "inferno", etc.).

\item[\code{node\_size\_range}] Numeric vector of length 2 specifying the range for node sizes.
Default is c(3, 12).

\item[\code{edge\_width\_range}] Numeric vector of length 2 specifying the range for edge widths.
Default is c(0.5, 3).

\item[\code{title}] Character string for plot title. Default is NULL.

\item[\code{...}] Additional arguments passed to the ggraph layout function.
\end{ldescription}
\end{Arguments}
%
\begin{Details}
Ego networks are particularly useful for understanding the mobility patterns
of specific social positions. The visualization highlights:
\begin{itemize}

\item{} The focal position (ego) in a distinct color
\item{} Incoming mobility flows (edges pointing to ego)
\item{} Outgoing mobility flows (edges from ego)
\item{} The relative strength of different flows through edge width and color

\end{itemize}


Only non-zero mobility flows are displayed to reduce visual clutter.
Edge colors and widths are scaled to represent the volume of mobility flows.
\end{Details}
%
\begin{Value}
A ggplot2 object showing the ego network visualization.
\end{Value}
%
\begin{SeeAlso}
\code{\LinkA{plot\_moneca\_ggraph}{plot.Rul.moneca.Rul.ggraph}} for full network visualization,
\code{\LinkA{plot\_stair\_ggraph}{plot.Rul.stair.Rul.ggraph}} for multi-level visualization,
\code{\LinkA{moneca}{moneca}} for the main analysis function
\end{SeeAlso}
%
\begin{Examples}
\begin{ExampleCode}
# Generate synthetic data and run MONECA
mobility_data <- generate_mobility_data(n_classes = 6, seed = 123)
seg <- moneca(mobility_data, segment.levels = 3)

# Ego network for the middle category (index 3)
plot_ego_ggraph(seg, mobility_data, ego_id = 3, 
                title = "Mobility from Middle Class")

# Ego network using category name (if available)
if (!is.null(rownames(mobility_data))) {
  plot_ego_ggraph(seg, mobility_data, ego_id = rownames(mobility_data)[1])
}

# Focus on strong mobility flows only (weight >= 10)
plot_ego_ggraph(seg, mobility_data, 
                ego_id = 2,
                min_weight = 10,
                title = "Strong Mobility Flows")

# Customized ego plot
plot_ego_ggraph(seg, mobility_data, 
                ego_id = 2,
                layout = "fr",
                highlight_color = "orange",
                flow_color = "plasma",
                edge_width_range = c(1, 5),
                title = "Professional Class Mobility")

\end{ExampleCode}
\end{Examples}
\HeaderA{plot\_moneca\_ggraph}{Modern Network Visualization for MONECA Results}{plot.Rul.moneca.Rul.ggraph}
%
\begin{Description}
Creates sophisticated network visualizations of MONECA clustering results using
ggraph and ggplot2. This function provides modern, highly customizable plots
with support for multiple layout algorithms, node aesthetics, and segment highlighting.
\end{Description}
%
\begin{Usage}
\begin{verbatim}
plot_moneca_ggraph(
  segments,
  level = NULL,
  layout = "fr",
  edges = "auto",
  node_size = "total",
  node_color = "segment",
  node_alpha = 0.8,
  edge_width = "weight",
  edge_color = "grey50",
  edge_alpha = 0.6,
  show_labels = TRUE,
  label_size = 3,
  show_segments = TRUE,
  segment_alpha = 0.3,
  color_palette = "Set3",
  theme_style = "void",
  title = NULL,
  segment_naming = "auto",
  ...
)
\end{verbatim}
\end{Usage}
%
\begin{Arguments}
\begin{ldescription}
\item[\code{segments}] A MONECA object returned by \code{\LinkA{moneca}{moneca}}.

\item[\code{level}] Integer vector specifying which hierarchical levels to visualize.
Default displays all levels except the first (which represents individual categories).

\item[\code{layout}] Character string or matrix specifying the layout algorithm:
\begin{itemize}

\item{} "fr" (default): Fruchterman-Reingold force-directed layout
\item{} "kk": Kamada-Kawai layout
\item{} "dh": Davidson-Harel layout  
\item{} "mds": Multidimensional scaling
\item{} "stress": Stress majorization
\item{} Matrix: Custom coordinate matrix (n\_nodes x 2)

\end{itemize}


\item[\code{edges}] Edge matrix or "auto" to automatically generate using 
\code{\LinkA{segment.edges}{segment.edges}}. Default is "auto".

\item[\code{node\_size}] Aesthetic mapping for node size:
\begin{itemize}

\item{} "total": Size by total mobility volume (default)
\item{} "mobility": Size by off-diagonal mobility rate
\item{} Numeric vector: Custom sizes for each node
\item{} Single numeric: Fixed size for all nodes

\end{itemize}


\item[\code{node\_color}] Aesthetic mapping for node color:
\begin{itemize}

\item{} "segment" (default): Color by segment membership
\item{} "mobility": Color by mobility rate
\item{} Character vector: Custom colors for each node
\item{} Single color: Fixed color for all nodes

\end{itemize}


\item[\code{node\_alpha}] Numeric value (0-1) for node transparency. Default is 0.8.

\item[\code{edge\_width}] Aesthetic for edge width:
\begin{itemize}

\item{} "weight" (default): Width proportional to edge weight
\item{} Numeric: Fixed width for all edges

\end{itemize}


\item[\code{edge\_color}] Color for edges. Default is "grey50".

\item[\code{edge\_alpha}] Numeric value (0-1) for edge transparency. Default is 0.6.

\item[\code{show\_labels}] Logical indicating whether to display node labels. Default is TRUE.

\item[\code{label\_size}] Numeric size for node labels. Default is 3.

\item[\code{show\_segments}] Logical indicating whether to highlight segment boundaries.
Default is TRUE.

\item[\code{segment\_alpha}] Numeric value (0-1) for segment boundary transparency.
Default is 0.3.

\item[\code{color\_palette}] Character string specifying the color palette for segments.
Can be any RColorBrewer palette name. Default is "Set3".

\item[\code{theme\_style}] Character string specifying the plot theme:
\begin{itemize}

\item{} "void" (default): Clean background with no axes
\item{} "minimal": Minimal theme with subtle gridlines
\item{} "classic": Traditional ggplot2 theme

\end{itemize}


\item[\code{title}] Character string for plot title. Default is NULL (no title).

\item[\code{segment\_naming}] Character string specifying the naming strategy for 
segment labels. Options are "auto", "concat", "pattern", or "custom". 
Default is "auto". See \code{\LinkA{segment.membership.enhanced}{segment.membership.enhanced}} for details.

\item[\code{...}] Additional arguments passed to ggraph layout functions.
\end{ldescription}
\end{Arguments}
%
\begin{Details}
This function creates publication-quality network visualizations with extensive
customization options. It automatically handles node positioning, edge rendering,
and segment highlighting. The resulting plot can be further modified using
standard ggplot2 syntax.

For interactive exploration, different layout algorithms may work better with
different network structures. Force-directed layouts ("fr") work well for most
cases, while "stress" layouts often produce cleaner results for dense networks.
\end{Details}
%
\begin{Value}
A ggplot2 object that can be further customized or displayed.
\end{Value}
%
\begin{SeeAlso}
\code{\LinkA{moneca}{moneca}} for the main analysis function,
\code{\LinkA{plot\_ego\_ggraph}{plot.Rul.ego.Rul.ggraph}} for ego network visualization,
\code{\LinkA{plot\_stair\_ggraph}{plot.Rul.stair.Rul.ggraph}} for multi-level visualization,
\code{\LinkA{segment.edges}{segment.edges}} for edge matrix generation
\end{SeeAlso}
%
\begin{Examples}
\begin{ExampleCode}
# Generate synthetic data and run MONECA
mobility_data <- generate_mobility_data(n_classes = 6, seed = 123)
seg <- moneca(mobility_data, segment.levels = 3)

# Basic network plot
plot_moneca_ggraph(seg)

# Customized plot with different aesthetics
plot_moneca_ggraph(seg,
  layout = "stress",
  node_color = "mobility",
  edge_width = "weight",
  color_palette = "Spectral",
  title = "Social Mobility Network",
  show_segments = FALSE
)

# Plot with custom node sizes and colors
custom_plot <- plot_moneca_ggraph(seg,
  node_size = c(8, 6, 10, 4, 7, 5),
  node_color = "red",
  edge_color = "darkblue",
  theme_style = "minimal"
)

# Further customize with ggplot2
custom_plot + 
  ggplot2::labs(subtitle = "Custom subtitle") +
  ggplot2::theme(plot.title = ggplot2::element_text(size = 16))

\end{ExampleCode}
\end{Examples}
\HeaderA{plot\_stair\_ggraph}{Multi-Level Segmentation Visualization (Stair Plot)}{plot.Rul.stair.Rul.ggraph}
%
\begin{Description}
Creates a series of network plots showing how segmentation evolves across
hierarchical levels in a MONECA analysis. This "stair plot" provides insight
into the progressive clustering of social positions.
\end{Description}
%
\begin{Usage}
\begin{verbatim}
plot_stair_ggraph(
  segments,
  levels = seq_along(segments$segment.list)[-1],
  layout = NULL,
  ncol = 2,
  segment_naming = "auto",
  ...
)
\end{verbatim}
\end{Usage}
%
\begin{Arguments}
\begin{ldescription}
\item[\code{segments}] A MONECA object returned by \code{\LinkA{moneca}{moneca}}.

\item[\code{levels}] Integer vector specifying which hierarchical levels to visualize.
Default includes all levels except the first (individual categories).

\item[\code{layout}] Layout specification for consistency across plots. Can be:
\begin{itemize}

\item{} NULL (default): Use layout.matrix() for consistent positioning
\item{} Character string: Layout algorithm name ("fr", "kk", "stress", etc.)
\item{} Matrix: Custom coordinate matrix for node positions

\end{itemize}


\item[\code{ncol}] Integer specifying the number of columns in the plot grid.
Default is 2. Set to 1 for vertical arrangement.

\item[\code{...}] Additional arguments passed to \code{\LinkA{plot\_moneca\_ggraph}{plot.Rul.moneca.Rul.ggraph}}.

\item[\code{combine\_plots}] Logical indicating whether to combine plots into a single
grid (TRUE) or return a list of individual plots (FALSE). Default is TRUE.
\end{ldescription}
\end{Arguments}
%
\begin{Details}
The stair plot helps visualize the hierarchical nature of MONECA segmentation
by showing how larger segments at higher levels break down into smaller, more
specific segments at lower levels. This is particularly useful for:
\begin{itemize}

\item{} Understanding the segmentation process
\item{} Identifying optimal levels of analysis
\item{} Presenting results to different audiences
\item{} Comparing segmentation stability across levels

\end{itemize}


When using a consistent layout across all plots, the relative positions of
nodes remain the same, making it easier to track how segments evolve.
\end{Details}
%
\begin{Value}
If \code{combine\_plots = TRUE}, returns a combined plot grid object.
If \code{combine\_plots = FALSE}, returns a list of ggplot objects, one for each level.
\end{Value}
%
\begin{SeeAlso}
\code{\LinkA{plot\_moneca\_ggraph}{plot.Rul.moneca.Rul.ggraph}} for single-level visualization,
\code{\LinkA{plot\_ego\_ggraph}{plot.Rul.ego.Rul.ggraph}} for ego network analysis,
\code{\LinkA{layout.matrix}{layout.matrix}} for consistent layouts,
\code{\LinkA{moneca}{moneca}} for the main analysis function
\end{SeeAlso}
%
\begin{Examples}
\begin{ExampleCode}
# Generate synthetic data and run MONECA
mobility_data <- generate_mobility_data(n_classes = 6, seed = 123)
seg <- moneca(mobility_data, segment.levels = 4)

# Basic stair plot
stair_plots <- plot_stair_ggraph(seg)

# Customized stair plot with specific levels
custom_stair <- plot_stair_ggraph(seg, 
                                 levels = c(2, 3),
                                 layout = "stress",
                                 ncol = 1,
                                 node_color = "mobility")

# Return individual plots for further customization
plot_list <- plot_stair_ggraph(seg, combine_plots = FALSE)
# Modify individual plots
plot_list[[1]] <- plot_list[[1]] + ggplot2::labs(subtitle = "Level 2")

## Not run: 
# Display the combined plot
print(stair_plots)

# Save individual plots
for (i in seq_along(plot_list)) {
  ggplot2::ggsave(paste0("level_", i, ".png"), plot_list[[i]])
}

## End(Not run)

\end{ExampleCode}
\end{Examples}
\HeaderA{print.descriptive.moneca}{Print Method for Descriptive MONECA Statistics}{print.descriptive.moneca}
\keyword{internal}{print.descriptive.moneca}
%
\begin{Description}
Internal function that formats and displays the detailed statistics calculated
by the \code{print.moneca} method. This function creates formatted tables showing
network properties and mobility statistics across hierarchical levels.
\end{Description}
%
\begin{Usage}
\begin{verbatim}
## S3 method for class 'descriptive.moneca'
print(x, ...)
\end{verbatim}
\end{Usage}
%
\begin{Arguments}
\begin{ldescription}
\item[\code{x}] A descriptive.moneca object containing calculated statistics.

\item[\code{...}] Additional arguments (currently unused).
\end{ldescription}
\end{Arguments}
%
\begin{Details}
This internal function is responsible for the formatted display of MONECA
analysis results. It creates several summary tables:
\begin{itemize}

\item{} Degree distributions (all, in, out) for each level
\item{} Edge weight distributions for each level
\item{} Diagonal mobility percentages
\item{} Network density and connectivity statistics

\end{itemize}

\end{Details}
%
\begin{Value}
NULL (called for side effects - printing formatted output).
\end{Value}
\HeaderA{print.first\_level\_summary}{Print first level summary}{print.first.Rul.level.Rul.summary}
%
\begin{Description}
Print first level summary
\end{Description}
%
\begin{Usage}
\begin{verbatim}
## S3 method for class 'first_level_summary'
print(x, ...)
\end{verbatim}
\end{Usage}
\HeaderA{print.moneca}{Print Method for MONECA Objects}{print.moneca}
%
\begin{Description}
Provides a comprehensive summary of MONECA analysis results including mobility
statistics, network properties, and segmentation quality measures across all
hierarchical levels.
\end{Description}
%
\begin{Usage}
\begin{verbatim}
## S3 method for class 'moneca'
print(x, small.cell.reduction = x$small.cell.reduction, ...)
\end{verbatim}
\end{Usage}
%
\begin{Arguments}
\begin{ldescription}
\item[\code{x}] A MONECA object returned by \code{\LinkA{moneca}{moneca}}.

\item[\code{small.cell.reduction}] Numeric threshold for small cell handling. If NULL,
uses the value from the MONECA object.

\item[\code{...}] Additional arguments (currently unused).
\end{ldescription}
\end{Arguments}
%
\begin{Details}
The print method calculates and displays several key statistics:

\strong{Mobility Statistics:}
\begin{itemize}

\item{} Total mobility rate (proportion of off-diagonal movement)
\item{} Diagonal mobility (immobility) by level
\item{} Mobility captured by significant edges

\end{itemize}


\strong{Network Properties:}
\begin{itemize}

\item{} Node degrees (in, out, total) by level
\item{} Network density by level
\item{} Number of isolated nodes
\item{} Edge weight distributions

\end{itemize}


\strong{Segmentation Quality:}
\begin{itemize}

\item{} Number of segments per level
\item{} Proportion of mobility within vs. between segments
\item{} Network coherence measures

\end{itemize}

\end{Details}
%
\begin{Value}
Invisibly returns a list of descriptive statistics, but primarily called
for its side effect of printing a formatted summary.
\end{Value}
%
\begin{SeeAlso}
\code{\LinkA{moneca}{moneca}}, \code{\LinkA{summary.moneca}{summary.moneca}}
\end{SeeAlso}
%
\begin{Examples}
\begin{ExampleCode}
# Generate data and run analysis
mobility_data <- generate_mobility_data(n_classes = 6, seed = 42)
seg <- moneca(mobility_data, segment.levels = 3)

# Print comprehensive summary
print(seg)

# The summary includes mobility rates, network statistics, and
# segmentation quality measures for each hierarchical level

\end{ExampleCode}
\end{Examples}
\HeaderA{segment.colors}{Generate Colors for Segments}{segment.colors}
%
\begin{Description}
Creates a grayscale color scheme for MONECA segments based on internal
mobility rates. Darker colors indicate higher immobility (lower internal mobility).
\end{Description}
%
\begin{Usage}
\begin{verbatim}
segment.colors(segments)
\end{verbatim}
\end{Usage}
%
\begin{Arguments}
\begin{ldescription}
\item[\code{segments}] A MONECA object returned by \code{\LinkA{moneca}{moneca}}.
\end{ldescription}
\end{Arguments}
%
\begin{Details}
This function calculates grayscale colors where the intensity reflects
the immobility rate within each segment. Segments with higher immobility
(more stable positions) receive darker colors.
\end{Details}
%
\begin{Value}
A list of color vectors, one for each hierarchical level.
\end{Value}
\HeaderA{segment.edges}{Extract Segment Edge Matrix}{segment.edges}
%
\begin{Description}
Creates an adjacency matrix representing edges between segments based on
mobility flows. This function is used for network visualization and analysis.
\end{Description}
%
\begin{Usage}
\begin{verbatim}
segment.edges(
  segments,
  cut.off = 1,
  mode = "directed",
  level = NULL,
  segment.reduction = NULL,
  method = "all",
  top = 3,
  diagonal = NULL,
  small.cell.reduction = 0
)
\end{verbatim}
\end{Usage}
%
\begin{Arguments}
\begin{ldescription}
\item[\code{segments}] A MONECA object returned by \code{\LinkA{moneca}{moneca}}.

\item[\code{cut.off}] Numeric threshold for minimum relative risk to include an edge.
Default is 1.

\item[\code{mode}] Character string specifying the graph mode ("directed" or "undirected").
Default is "directed".

\item[\code{level}] Integer vector specifying which hierarchical levels to include.
Default includes all levels.

\item[\code{segment.reduction}] Integer vector specifying levels for which to remove
internal segment edges. Default includes all levels.

\item[\code{method}] Character string specifying edge filtering method:
\begin{itemize}

\item{} "all" (default): Include all edges above threshold
\item{} "top.out": Keep only the top outgoing edges per node
\item{} "top.in": Keep only the top incoming edges per node

\end{itemize}


\item[\code{top}] Integer specifying how many top edges to keep when using 
"top.out" or "top.in" methods. Default is 3.

\item[\code{diagonal}] Controls diagonal values. If NULL (default), diagonal is zeroed.

\item[\code{small.cell.reduction}] Numeric threshold for small cell handling.
\end{ldescription}
\end{Arguments}
%
\begin{Value}
A square matrix representing edge weights between nodes/segments.
\end{Value}
%
\begin{SeeAlso}
\code{\LinkA{plot\_moneca\_ggraph}{plot.Rul.moneca.Rul.ggraph}}, \code{\LinkA{moneca.plot}{moneca.plot}}
\end{SeeAlso}
\HeaderA{segment.membership}{Extract Segment Membership Information}{segment.membership}
%
\begin{Description}
Returns a data frame showing which segment each original category belongs to
across the specified hierarchical levels of a MONECA analysis.
\end{Description}
%
\begin{Usage}
\begin{verbatim}
segment.membership(segments, level = seq(segments$segment.list))
\end{verbatim}
\end{Usage}
%
\begin{Arguments}
\begin{ldescription}
\item[\code{segments}] A MONECA object returned by \code{\LinkA{moneca}{moneca}}.

\item[\code{level}] Integer vector specifying which hierarchical levels to include.
Default includes all available levels.
\end{ldescription}
\end{Arguments}
%
\begin{Details}
The membership strings indicate both the hierarchical level and the specific
segment within that level. For example, "2.3" means the category belongs to
segment 3 at level 2 of the hierarchy.
\end{Details}
%
\begin{Value}
A data frame with two columns:
\begin{description}

\item[name] Character vector of original category names
\item[membership] Character vector indicating segment membership, formatted
as "level.segment" (e.g., "2.1" for level 2, segment 1)

\end{description}

\end{Value}
%
\begin{SeeAlso}
\code{\LinkA{moneca}{moneca}}, \code{\LinkA{plot\_moneca\_ggraph}{plot.Rul.moneca.Rul.ggraph}}
\end{SeeAlso}
%
\begin{Examples}
\begin{ExampleCode}
# Generate data and run analysis
mob_data <- generate_mobility_data(n_classes = 5, seed = 42)
seg <- moneca(mob_data, segment.levels = 3)

# Get membership information
membership <- segment.membership(seg)
print(membership)

# Get membership for specific levels only
membership_level2 <- segment.membership(seg, level = 2)

\end{ExampleCode}
\end{Examples}
\HeaderA{segment.membership.enhanced}{Enhanced Segment Membership with Meaningful Names}{segment.membership.enhanced}
%
\begin{Description}
Returns segment membership information with an additional column containing 
meaningful segment names derived from the constituent nodes. This function 
extends \code{\LinkA{segment.membership}{segment.membership}} by adding intelligent naming 
strategies for aggregated segments.
\end{Description}
%
\begin{Usage}
\begin{verbatim}
segment.membership.enhanced(
  segments,
  level = seq(segments$segment.list),
  naming_strategy = "auto",
  custom_names = NULL,
  separator = " + ",
  max_concat_length = 2,
  pattern_rules = NULL
)
\end{verbatim}
\end{Usage}
%
\begin{Arguments}
\begin{ldescription}
\item[\code{segments}] A MONECA object returned by \code{\LinkA{moneca}{moneca}}.

\item[\code{level}] Integer vector specifying which hierarchical levels to include.
Default includes all available levels.

\item[\code{naming\_strategy}] Character string specifying the naming approach:
\begin{itemize}

\item{} "auto" (default): Automatic naming based on segment composition
\item{} "concat": Concatenate node names with separator
\item{} "pattern": Use pattern recognition for common suffixes/prefixes
\item{} "custom": Use user-provided names from custom\_names parameter

\end{itemize}


\item[\code{custom\_names}] Named list providing custom names for specific segments.
Format: list("level.segment" = "Custom Name"), e.g., list("2.1" = "Blue Collar")

\item[\code{separator}] Character string used to separate names when concatenating.
Default is " + ".

\item[\code{max\_concat\_length}] Maximum number of names to concatenate before 
switching to pattern-based naming. Default is 2.

\item[\code{pattern\_rules}] Named list of pattern rules for automatic naming.
Default includes common patterns like "Worker" → "Workers\_Group".
\end{ldescription}
\end{Arguments}
%
\begin{Details}
The function applies different naming strategies based on segment composition:
\begin{enumerate}

\item{} \strong{Individual nodes}: Keep original name unchanged
\item{} \strong{Small segments} (≤ max\_concat\_length): Concatenate names
\item{} \strong{Large segments}: Apply pattern recognition or use generic names
\item{} \strong{Custom names}: Override automatic naming with user-provided names

\end{enumerate}


Pattern recognition identifies common suffixes (e.g., "Worker", "Manager") 
and creates group names (e.g., "Workers\_Group", "Management\_Group").
\end{Details}
%
\begin{Value}
A data frame with three columns:
\begin{description}

\item[name] Character vector of original category names
\item[membership] Character vector indicating segment membership, formatted
as "level.segment" (e.g., "2.1" for level 2, segment 1)
\item[level\_name] Character vector with meaningful segment names derived
from constituent nodes or user-provided names

\end{description}

\end{Value}
%
\begin{SeeAlso}
\code{\LinkA{segment.membership}{segment.membership}} for basic membership information,
\code{\LinkA{moneca}{moneca}} for the main analysis function
\end{SeeAlso}
%
\begin{Examples}
\begin{ExampleCode}
# Generate data and run analysis
mob_data <- generate_mobility_data(n_classes = 6, seed = 42,
  class_names = c("Upper_Class", "Professional", "Manager", 
                  "Skilled_Worker", "Service_Worker", "Manual_Worker"))
seg <- moneca(mob_data, segment.levels = 3)

# Get enhanced membership with automatic naming
enhanced <- segment.membership.enhanced(seg)
print(enhanced)

# Use concatenation strategy
concat_membership <- segment.membership.enhanced(seg, 
                                                naming_strategy = "concat")

# Use custom names for specific segments
custom_names <- list("2.1" = "Blue_Collar", "3.1" = "All_Classes")
custom_membership <- segment.membership.enhanced(seg, 
                                                naming_strategy = "custom",
                                                custom_names = custom_names)

\end{ExampleCode}
\end{Examples}
\HeaderA{segment.quality}{Segment quality}{segment.quality}
%
\begin{Description}
Segment quality
\end{Description}
%
\begin{Usage}
\begin{verbatim}
segment.quality(segments, final.solution = FALSE)
\end{verbatim}
\end{Usage}
\HeaderA{stair.plot}{Legacy Multi-Level Stair Plot}{stair.plot}
%
\begin{Description}
Creates a series of plots showing how segmentation evolves across hierarchical
levels using the legacy ggplot2 system. For modern stair plots, use
\code{\LinkA{plot\_stair\_ggraph}{plot.Rul.stair.Rul.ggraph}}.
\end{Description}
%
\begin{Usage}
\begin{verbatim}
stair.plot(
  segments,
  level = NULL,
  layout = NULL,
  edges = NULL,
  mode = "directed",
  vertex.size = "total",
  vertex.alpha = 1,
  vertex.color = "black",
  vertex.shape = 21,
  show.edges = TRUE,
  edge.size = 0.5,
  edge.alpha = "weight",
  edge.color = "black",
  edge.line = "solid",
  show.text = FALSE,
  text.size = 3,
  text.color = "black",
  text.alpha = 1,
  text.vjust = 1.5,
  show.borders = TRUE,
  border.size = 1,
  border.fill = NA,
  border.color = "black",
  border.alpha = 1,
  border.padding = 1,
  border.text = TRUE,
  border.labels = "segments",
  border.text.size = 4,
  border.text.color = "black",
  border.text.vjust = -0.2,
  border.text.hjust = 1,
  midpoints = TRUE,
  midpoint.arrow = arrow(angle = 20, length = unit(0.33, "cm"), ends = "last", type =
    "closed"),
  edge.text = FALSE,
  edge.text.size = 3,
  edge.text.alpha = 0.9,
  legend = "side",
  level.title = "Level"
)
\end{verbatim}
\end{Usage}
%
\begin{Arguments}
\begin{ldescription}
\item[\code{segments}] A MONECA object returned by \code{\LinkA{moneca}{moneca}}.

\item[\code{level}] Integer vector specifying which levels to include in the stair plot.

\item[\code{layout}] Layout matrix for consistent node positioning across plots.

\item[\code{edges}] Edge matrix or specification for network edges.

\item[\code{mode}] Character string specifying graph mode ("directed" or "undirected").

\item[\code{vertex.size}] Specification for vertex sizes ("total" or numeric).

\item[\code{vertex.alpha}] Numeric transparency for vertices (0-1).

\item[\code{vertex.color}] Color specification for vertex borders.

\item[\code{vertex.shape}] Numeric shape code for vertices.

\item[\code{show.edges}] Logical indicating whether to display edges.

\item[\code{edge.size}] Numeric size for edges.

\item[\code{edge.alpha}] Transparency for edges ("weight" or numeric).

\item[\code{edge.color}] Color specification for edges.

\item[\code{edge.line}] Line type for edges ("solid", "dashed", etc.).

\item[\code{show.text}] Logical indicating whether to show vertex labels.

\item[\code{text.size}] Numeric size for text labels.

\item[\code{text.color}] Color for text labels.

\item[\code{text.alpha}] Transparency for text labels.

\item[\code{text.vjust}] Vertical adjustment for text labels.

\item[\code{show.borders}] Logical indicating whether to show segment boundaries.

\item[\code{border.size}] Size for segment borders.

\item[\code{border.fill}] Fill color for segment boundaries.

\item[\code{border.color}] Color for segment border lines.

\item[\code{border.alpha}] Transparency for segment borders.

\item[\code{border.padding}] Padding around segment boundaries.

\item[\code{border.text}] Logical indicating whether to show segment labels.

\item[\code{border.labels}] Specification for segment labels.

\item[\code{border.text.size}] Size for segment labels.

\item[\code{border.text.color}] Color for segment labels.

\item[\code{border.text.vjust}] Vertical adjustment for segment labels.

\item[\code{border.text.hjust}] Horizontal adjustment for segment labels.

\item[\code{midpoints}] Logical indicating whether to show edge midpoints.

\item[\code{midpoint.arrow}] Logical indicating whether to show arrows at midpoints.

\item[\code{edge.text}] Logical indicating whether to show edge labels.

\item[\code{edge.text.size}] Size for edge labels.

\item[\code{edge.text.alpha}] Transparency for edge labels.

\item[\code{legend}] Position specification for legend.

\item[\code{level.title}] Specification for level titles.
\end{ldescription}
\end{Arguments}
%
\begin{Details}
This function creates multiple plots showing the progression of segmentation
across hierarchical levels. Each plot uses the same layout to maintain
consistency, making it easy to see how segments merge or split across levels.

\strong{Note}: This function is maintained for backward compatibility and
requires the eliter package. For new analyses, use \code{\LinkA{plot\_stair\_ggraph}{plot.Rul.stair.Rul.ggraph}}
which offers better performance and modern styling.
\end{Details}
%
\begin{Value}
A list of ggplot2 objects, one for each segmentation level.
\end{Value}
%
\begin{SeeAlso}
\code{\LinkA{plot\_stair\_ggraph}{plot.Rul.stair.Rul.ggraph}} for modern stair plots,
\code{\LinkA{gg.moneca}{gg.moneca}} for the underlying plotting function
\end{SeeAlso}
%
\begin{Examples}
\begin{ExampleCode}
## Not run: 
# Requires legacy data and eliter package
data(occupations)
plots <- stair.plot(mob.seg)
plots[[2]]  # Display second level

# Customized stair plot
custom_stairs <- stair.plot(mob.seg,
                           level = c(2, 3, 4),
                           vertex.size = "total",
                           show.borders = TRUE,
                           border.text.size = 5)

## End(Not run)

\end{ExampleCode}
\end{Examples}
\HeaderA{test\_igraph\_compatibility}{Test framework for dependency updates}{test.Rul.igraph.Rul.compatibility}
\keyword{Test}{test\_igraph\_compatibility}
\keyword{compatibility}{test\_igraph\_compatibility}
\keyword{igraph}{test\_igraph\_compatibility}
\keyword{internal}{test\_igraph\_compatibility}
\keyword{layer}{test\_igraph\_compatibility}
%
\begin{Description}
This script provides comprehensive testing for the igraph migration
and dependency updates to ensure compatibility and functionality
\end{Description}
%
\begin{Usage}
\begin{verbatim}
test_igraph_compatibility()
\end{verbatim}
\end{Usage}
%
\begin{Value}
List with test results
\end{Value}
\HeaderA{validate\_description\_dependencies}{Validate DESCRIPTION file dependencies}{validate.Rul.description.Rul.dependencies}
%
\begin{Description}
Validate DESCRIPTION file dependencies
\end{Description}
%
\begin{Usage}
\begin{verbatim}
validate_description_dependencies()
\end{verbatim}
\end{Usage}
%
\begin{Value}
List with validation results
\end{Value}
\HeaderA{vertex.mobility}{Vertex mobility}{vertex.mobility}
%
\begin{Description}
Vertex mobility
\end{Description}
%
\begin{Usage}
\begin{verbatim}
vertex.mobility(segments)
\end{verbatim}
\end{Usage}
\HeaderA{weight.matrix}{Calculate Relative Risk Weight Matrix}{weight.matrix}
%
\begin{Description}
Converts a mobility table into a relative risk matrix by comparing observed
mobility flows to expected flows under independence. This matrix forms the
basis for network construction in MONECA analysis.
\end{Description}
%
\begin{Usage}
\begin{verbatim}
weight.matrix(
  mx,
  cut.off = 1,
  symmetric = TRUE,
  diagonal = NULL,
  small.cell.reduction = 0
)
\end{verbatim}
\end{Usage}
%
\begin{Arguments}
\begin{ldescription}
\item[\code{mx}] A mobility table (square matrix) with row and column totals in the 
last row/column.

\item[\code{cut.off}] Numeric threshold for minimum relative risk. Values below this
are set to NA. Default is 1.

\item[\code{symmetric}] Logical indicating whether to force the matrix to be symmetric
by adding it to its transpose. Default is TRUE.

\item[\code{diagonal}] Controls diagonal values. If NULL (default), diagonal is set 
to NA. Otherwise, diagonal values are preserved.

\item[\code{small.cell.reduction}] Numeric value for handling small cells. Cells with
counts below this threshold are set to 0 before calculating relative risks.
Default is 0.
\end{ldescription}
\end{Arguments}
%
\begin{Details}
The relative risk for cell (i,j) is calculated as:
\deqn{RR_{ij} = O_{ij} / E_{ij}}{}
where \eqn{O_{ij}}{} is the observed count and \eqn{E_{ij}}{} is the expected
count under independence: \eqn{E_{ij} = (n_i * n_j) / N}{}
\end{Details}
%
\begin{Value}
A matrix of relative risks where:
\begin{itemize}

\item{} Values > 1 indicate mobility above expected levels
\item{} Values < 1 indicate mobility below expected levels  
\item{} Values below cut.off are set to NA

\end{itemize}

\end{Value}
%
\begin{SeeAlso}
\code{\LinkA{moneca}{moneca}} for the main analysis function
\end{SeeAlso}
%
\begin{Examples}
\begin{ExampleCode}
# Create a simple mobility table
mob_table <- matrix(c(100, 20, 10, 130,
                      15, 80, 25, 120,  
                      5,  10, 50,  65,
                      120, 110, 85, 315), 
                    nrow = 4, byrow = TRUE)
rownames(mob_table) <- colnames(mob_table) <- c("A", "B", "C", "Total")

# Calculate relative risk matrix
rr_matrix <- weight.matrix(mob_table, cut.off = 1.5)

\end{ExampleCode}
\end{Examples}
\printindex{}
\end{document}
